\documentclass[11pt, spanish]{article}
\usepackage[spanish]{babel}
\selectlanguage{spanish}
\usepackage{natbib}
\usepackage{url}
\usepackage[utf8x]{inputenc}
\usepackage{graphicx}
\graphicspath{{images/}}
\usepackage{parskip}
\usepackage{fancyhdr}
\usepackage{vmargin}
\usepackage{listings}
\usepackage{booktabs}
\usepackage[table,xcdraw]{xcolor}
\lstset{  
    mathescape=true
}

\usepackage{amsmath}%
\usepackage{MnSymbol}%
\usepackage{wasysym}%

\usepackage[default]{sourcesanspro}

\setmarginsrb{2 cm}{1 cm}{2 cm}{2 cm}{1 cm}{1.5 cm}{1 cm}{1.5 cm}

\title{Práctica 3 - Descripción Arquitectónica\\App catálogo de videojuegos\hspace{0.05cm}}
\author{Jose Luis Gallego Peña\\Sergio Campos Megías}
\date{\today}                                           

\renewcommand*\contentsname{hola}

\makeatletter
\let\thetitle\@title
\let\theauthor\@author
\let\thedate\@date
\makeatother

\pagestyle{fancy}
\fancyhf{}
\rhead{\theauthor}
\lhead{\thetitle}
\cfoot{\thepage}

\begin{document}
%%%%%%%%%%%%%%%%%%%%%%%%%%%%%%%%%%%%%%%%%%%%%%%%%%%%%%%%%%%%%%%%%%%%%%%%%%%%%%%%%%%%%%%%%

\begin{titlepage}
    \centering
    \vspace*{0.1 cm}
    \includegraphics[scale = 0.50]{ugr.png}\\[0.5 cm]
    %\textsc{\LARGE Universidad de Granada}\\[2.0 cm]   
    \textsc{\large 3º Ingeniería del Software}\\[0.5 cm]        
    \textsc{\large Desarrollo de Software}\\[0.5 cm]       
    \textsc{\large Grado en Ingeniería Informática}\\[0.5 cm]              
    \rule{\linewidth}{0.2 mm} \\[0.2 cm]
    { \huge \bfseries \thetitle}\\
    \rule{\linewidth}{0.2 mm} \\[1.5 cm]
    	
    \begin{minipage}{1\textwidth}
        \begin{center} \large
            \theauthor
        \end{center}
    \end{minipage}~
    
    \vspace{2cm}
    {\large \thedate}\\[1 cm]
 	
    \vfill
    
\end{titlepage}

%%%%%%%%%%%%%%%%%%%%%%%%%%%%%%%%%%%%%%%%%%%%%%%%%%%%%%%%%%%%%%%%%%%%%%%%%%%%%%%%%%%%%%%%%

\tableofcontents
\pagebreak

%%%%%%%%%%%%%%%%%%%%%%%%%%%%%%%%%%%%%%%%%%%%%%%%%%%%%%%%%%%%%%%%%%%%%%%%%%%%%%%%%%%%%%%%%
%\fbox{\includegraphics[scale=0.25]{.png}}
\section{Análisis de requisitos}

Se realizará la descripción arquitectónica de un sistema multiplataforma de gestión de catálogo de videojuegos. Se trata de un sistema que muestra información sobre distintos videojuegos y enlaces hacia páginas para comprarlos.

Sus principales objetivos son:

\begin{itemize}
\item Una app móvil para el usuario que interaccione con un servidor web REST desplegado en internet, con la cual se podrá acceder al catálogo de videojuegos, consultar información y redirigirse a las distintas páginas para comprarlo.

\item Una aplicación web REST desplegada en internet que permita a los administradores acceder al catálogo para añadir nuevos videojuegos, eliminarlos o modificar datos.

\item Establecer un sistema funcional y rápido con una interfaz de usuario simple e intuitiva para garantizar la fidelidad de los usuarios.
\end{itemize}

Los desglosamos a continuación en requisitos no funcionales y funcionales, con el posterior análisis de cada uno.

\subsection{Análisis de requisitos no funcionales}

\begin{itemize}
\item \textbf{RNF1} - El sistema web usará tecnologías REST y estará programado en el lenguaje Java con Jersey.

\item \textbf{RNF2} - Se usará el servidor web que proporciona el IDE Netbeans con Apache Tomcat.

\item \textbf{RNF3} - La aplicación estará desarrollada en Java Android usando Android Studio junto a la biblioteca Picasso.

\item \textbf{RNF4} - La interfaz de la aplicación será simple e intuitiva, con pocas opciones, siguiendo los estándares Android Material Design.

\item \textbf{RNF5} - La aplicación interaccionará con el servidor REST desplegado en internet usando la biblioteca Retrofit.

\item \textbf{RNF6} - Permitirá únicamente el castellano, aunque podrá redirigir a páginas de cualquier idioma. Tendrá la posibilidad de añadir nuevos idiomas en el futuro.

\item \textbf{RNF7} - Debe cumplir la ley de protección de datos y la privacidad de estos.
\end{itemize}

Además, el sistema debe mantener los siguientes criterios de calidad:

\begin{itemize}
\item \textbf{RNF8} - \textbf{Seguridad}: El sistema debe acceder de forma segura a los datos, asegurando al confidencialidad de estos datos.

\item \textbf{RNF9} - \textbf{Fiabilidad}: El sistema deberá funcionar sin fallos y responder a las peticiones de forma rápida y correcta.

\item \textbf{RNF10} - \textbf{Mantenibilidad}: El sistema deberá ser lo suficientemente simple como para poder mantenerse y usarse durante un periodo grande de tiempo, pudiendo además añadir nuevas funcionalidades con el tiempo de forma simple gracias a un buen diseño.
\end{itemize}

\subsection{Análisis de requisitos funcionales}

\begin{itemize}
\item \textbf{RF1} - Añadir videojuego al catálogo
\item \textbf{RF2} - Modificar descripción videojuego
\item \textbf{RF3} - Modificar precio videojuego
\item \textbf{RF4} - Modificar página videojuego
\item \textbf{RF5} - Modificar imagen videojuego
\item \textbf{RF6} - Eliminar videojuego del catálogo
\item \textbf{RF7} - Consultar información videojuego
\item \textbf{RF8} - Acceder a página del videojuego
\end{itemize}

Requisitos funcionales que NO son responsabilidad del sistema:

\begin{itemize}
\item Implementar la tecnología REST
\item Desarrollar el servidor web
\end{itemize}

\section{Punto de vista contextual}

El sistema consiste en dos grandes subsistemas:

\begin{itemize}
\item Para el usuario: Una app que interacciona con un servidor web, con la cual se podrá acceder al catálogo de videojuegos, consultar información y redirigirse a las distintas páginas para comprarlo.
\item Para los mantenedores/desarrolladores: Una aplicación web que permita a los administradores acceder al catálogo para añadir nuevos videojuegos, eliminarlos o modificar datos.
\end{itemize}

\begin{center}
\fbox{\includegraphics[scale=0.65]{contexto.png}}
\end{center}

\section{Diagramas de clases de diseño}

\begin{center}
\fbox{\includegraphics[scale=0.26]{diagrama_app.png}}
\end{center}

\section{Memoria}

Para el desarrollo de la aplicación android se optó por usar Android Studio puesto que es uno de las principales herramientas que buscábamos aprender con esta práctica. La dificultad de su uso nos ha hecho tener muchos problemas, entre ellos primeramente el crear el proyecto, ya que daba muchos errores extraños que tuvimos que solucionar mirando por internet, como por ejemplo instalar dependencias del IDE que faltaban.

Gracias a las herramientas de generación automática del IDE pudimos crear la estructura básica del programa y otros métodos como getters y setters fácilmente.

El IDE deja crear una plantilla de proyecto, pero nosotros la editamos fuertemente después de estar un día mirando cómo estaban conectadas las distintas clases y cual era la estructura de funcionamiento de una app android. Hasta aquí todo bien, puesto que era mirar soluciones muy fáciles en un lenguaje, Java, que ya conocemos.

Sin embargo surgieron problemas al empezar a desarrollar la aplicación que queríamos, dos principalmente. El primero, con respecto a representar los datos. Usamos un tipo ListView para representar el catálogo, y esta es una estructura de datos bastante compleja de usar y que a la mínima de cambio no funciona, por lo que estuvimos bastante tiempo intentando conseguir que se mostrase lo que queríamos y de la forma que queríamos, teniendo que tocar los archivos XML de la interfaz gráfica ya que con el asistente no podíamos.

El segundo y mayor problema fue conectar la aplicación a un servicio API Rest ya que hay que realizar muchos pasos. En el archivo manifiesto de android hay que especificar que la app se conecta a internet, pero además es que hay muchas formas de resolver esto y todas muy complejas. Nos decidimos por aprender a usar la biblioteca Retrofit porque parecía la mejor opción, y lo es, sin embargo estuvimos un día entero para solucionar la conexión ya que nos daba error debido a que al tener un servicio en localhost, no se puede usar el DNS "localhost" si no que hay que usar una IP especial: 10.0.2.2. Además de esto, fue difícil porque esta biblioteca requiere crear una nueva clase, Post, en la que almacenar exactamente (mismo nombre y tipo) los datos que recibiremos de un JSON de la API Rest.

En general, Android Studio es un IDE bastante complejo de usar, con muchísimas particularidades, y sobre todo, el mayor problema que tiene es que está fatal optimizado y tarda muchísimo en compilar y funcionar fluidamente todo. Hemos llegado a tener que reiniciar el ordenador mas de 5 veces.

\end{document}